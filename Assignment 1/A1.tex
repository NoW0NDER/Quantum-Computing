\documentclass{article}
\usepackage{lipsum}
\usepackage{braket}
\usepackage{mathrsfs}
\begin{document}
\title{Assignment 1}
\author{
    Name:\underline{Wang Dingrui}
}
\maketitle
\section*{EXERCISE 1: COMPLEX NUMBERS}
\begin{enumerate}
    \item Assume $x=a_1+b_1i$, $x=a_2+b_2i$, $x=a_3+b_3i$,

          we have $x+y+z=a_1+a_2+a_3+(b_1+b_2+b_3)i$,

          To a complex number $n=a+bi$ we have $n^*=a-bi$,

          so

          $|x|^2=x\cdot x^*=a_1^2-b_1^2(i)^2=a_1^2-b_1^2\cdot(-1)=a_1^2+b_1^2$,

          $x^*y \\= (a_1-b_1i)\cdot(a_2+b_2i)\\=a_1a_2+a_1b_2i-a_2b_1i-b_1b_2i^2\\=a_1a_2+b_1b_2+(a_1b_2-a_2b_1)i$,

          $Re(x^*y)=a_1a_2+b_1b_2$,

          similarly, we have

          $|y|^2=a_2^2+b_2^2$,

          $|z|^2=a_3^2+b_3^2$,

          $Re(y^*z)=a_2a_3+b_2b_3$

          $Re(x^*z)=a_1a_3+b_1b_3$

          and



          $(x+y+z)^*=a_1+a_2+a_3-(b_1+b_2+b_3)i$,

          $|x+y+z|^2\\=(x+y+z)\cdot(x+y+z)^*
              \\=((a_1+a_2+a_3)+(b_1+b_2+b_3)i)\cdot((a_1+a_2+a_3)-(b_1+b_2+b_3)i)
              \\=((a_1+a_2+a_3)^2-(b_1+b_2+b_3)^2\cdot(-1))
              \\=((a_1+a_2+a_3)^2+(b_1+b_2+b_3)^2)
              \\=a_1^2+a_2^2+a_3^2+2a_1a_2+2a_1a_3+2a_2a_3+b_1^2+b_2^2+b_3^2+2b_1b_2+2b_1b_3+2b_2b_3
              \\=a_1^2+b_1^2+a_2^2+b_2^2+a_3^2+b_3^2+2(a_1a_2+b_1b_2+a_2a_3+b_2b_3+a_1a_3+b_1b_3)
              \\=|x|^2+|y|^2+|z|^2+2[Re(x^*y)+Re(y^*z)+Re(x^*z)]
          $

          This shows that

          $|x+y+z|^2=|x|^2+|y|^2+|z|^2+2[Re(x^*y)+Re(y^*z)+Re(x^*z)]$
    \item $(i+2)(3-4i)/(2-i)
              \\=(3i-4i^2+2*3-2*4i)/(2-i)
              \\=(3i-4\times(-1)+2*3-2*4i)/(2-i)
              \\=(3i+4+6-8i)/(2-i)
              \\=(10-5i)/(2-i)
              \\=5(2-i)/(2-i)
              \\=5
          $
    \item $ (i-4)/(2i-3)
              \\=[(i-4)(2i+3)]/[(2i-3)(2i+3)]
              \\=(2i^2+3i-8i-4*3)/((2i)^2-3*3)
              \\=[2\times(-1)+3i-8i-4*3]/[4\times(-1)-3*3]
              \\=(-2-5i-12)/(-4-9)
              \\=(-14-5i)/(-13)
              \\=[(-1)(14+5i)]/(-1\times13)
              \\=(14+5i)/13
              \\=14/13+5/13i
          $

          so, the real part is $14/13$ and imaginary pary is $5/13$.
    \item $i^{33}
              \\=i^{32}i
              \\=i^{2\times16}i
              \\=(i^2)^{16}i
              \\=(-1)^{16}i
              \\=i
          $

          so, the absolute value of $i^{33}$ is $|i|$

          $|i| = |0+i| = \sqrt{|0+i|^2} = \sqrt{(0+i)(0+i)^*}=\sqrt{(0+i)(0-i)}=\sqrt{-i^2}=\sqrt{-(-1)}=\sqrt{1}=1$
    \item \begin{enumerate}
              \item[i.] For complex number $c_1=a_1+b_1i$ and $c_2=a_2+b_2i$, we have
                    \[|c_1|^2=a_1^2+b_1^2\]
                    \[|c_2|^2=a_2^2+b_2^2\]
                    \[|c_1+c2|^2=(a_1+a_2)^2+(b_1+b_2)^2
                        \\=a_1^2+a_2^2+2a_1a_2+b_1^2+b_2^2+2b_1b_2\]

                    so we need to find $a_1,a_2,b_1,b_2$ that makes
                    \[a_1^2+a_2^2+2a_1a_2+b_1^2+b_2^2+2b_1b_2\geq a_1^2+b_1^2\]
                    and
                    \[a_1^2+a_2^2+2a_1a_2+b_1^2+b_2^2+2b_1b_2<a_2^2+b_2^2\]
                    so we have
                    \[a_2^2+2a_1a_2+b_2^2+2b_1b_2\geq 0\]
                    and
                    \[a_1^2+2a_1a_2+b_1^2+2b_1b_2<0\]
                    so, we need $2a_1a_2+2b_1b_2\geq-(a_2^2+b_2^2)$ and $2a_1a_2+2b_1b_2<-(a_1^2+b_1^2)$,

                    which means $-(a_2^2+b_2^2)\leq 2a_1a_2+2b_1b_2<-(a_1^2+b_1^2)$,

                    Through observing, it is easy to find $a_1=-1,a_2=3,b_1=1,b2=-3$ makes $|c_1+c_2|^2\geq |c_1|^2$ and $|c_1+c_2|^2< |c_2|^2$.

              \item[ii.] To make $|c_1+c_2|^2< |c_1|^2$ and $|c_1+c_2|^2< |c_2|^2$,
                    we need to make
                    \[a_1^2+a_2^2+2a_1a_2+b_1^2+b_2^2+2b_1b_2< a_1^2+b_1^2\]
                    and
                    \[a_1^2+a_2^2+2a_1a_2+b_1^2+b_2^2+2b_1b_2< a_2^2+b_2^2\]
                    so, we have
                    \[a_2^2+2a_1a_2+b_2^2+2b_1b_2< 0\]
                    and
                    \[a_1^2+2a_1a_2+b_1^2+2b_1b_2< 0\]
                    I don't think we can find $a_1,a_2,b_1,b_2$ that can make $|c_1+c_2|^2< |c_1|^2$ and $|c_1+c_2|^2< |c_2|^2$ true.

          \end{enumerate}
    \item Assume both $\vec{v}_1$ and $\vec{v}_2$ has a length of $n$.

          $\vec{v}_1=(\psi_{10},\psi_{11},\psi_{12},\psi_{13},...,\psi_{1n})^T$,

          $\vec{v}_2=(\psi_{20},\psi_{21},\psi_{22},\psi_{23},...,\psi_{2n})^T$.

          For real vectors $\vec{r}_1$ and $\vec{r}_2$, we have $\langle \vec{r}_1,\vec{r}_2 \rangle=\vec{r}_1^T\vec{r}_2$.

          Similarly, we can define the inner product of $\vec{v}_1,\vec{v}_2$ that
          \[\langle \vec{v}_1,\vec{v}_2 \rangle
              \\=\vec{v}_1^T\vec{v}_2
          \]
          where $\vec{v}_1^T$ is the transpose of $\vec{v}_1$.

          This means
          \[
              \langle \vec{v}_1,\vec{v}_2 \rangle=\\=(\psi_{10},\psi_{11},\psi_{12},\psi_{13},...,\psi_{1n})(\psi_{20},\psi_{21},\psi_{22},\psi_{23},...,\psi_{2n})^T
          \]
          so,
          \[\langle \vec{v}_1,\vec{v}_2 \rangle=\sum_{i=0}^{n-1}\psi_{1i}\psi_{2i}\]
          The properties of an inner product $\langle,\rangle$ are as followed\cite{ref1}.
          \begin{enumerate}
              \item \textbf{Linearity:} $\langle a\mathbf{u}+b\mathbf{v},\mathbf{w}\rangle=a\langle \mathbf{u,w}\rangle+b\langle \mathbf{v,w}\rangle$
              \item \textbf{Symmetric Property: }$\langle \mathbf{u,v}\rangle=\langle \mathbf{v,u}\rangle$
              \item \textbf{Positive Definite Property: } For any $\mathbf{u\in V}$, $\langle \mathbf{u,u}\rangle\geq0$;
                    and $\langle \mathbf{u,u}\rangle=0$ if and only if $\mathbf{u}=0$;
          \end{enumerate}
          For complex vectors $\vec{v1},\vec{v_2},\vec{v_3}$, all of them have a length of n.

          $\langle a\vec{v_1}+b\vec{v_2},\vec{v_3}\rangle
              \\=\sum_{i=0}^{n-1}(a\psi_{1i}+b\psi_{2i})(\psi_{3i})
              \\=\sum_{i=0}^{n-1}(a\psi_{1i})(\psi_{3i})+\sum_{i=0}^{n-1}(b\psi_{2i})(\psi_{3i})
              \\=a\sum_{i=0}^{n-1}(\psi_{1i})(\psi_{3i})+b\sum_{i=0}^{n-1}(\psi_{2i})(\psi_{3i})
              \\=a\langle\vec{v_1},\vec{v_3}\rangle+b\langle\vec{v_2},\vec{v_3}\rangle
          $

          This proves the linearity.

          Also, we have

          $\langle \vec{v_1},\vec{v_2}\rangle
              \\=\sum_{i=0}^{n-1}\psi_{1i}\psi_{2i}
              \\=\sum_{i=0}^{n-1}\psi_{2i}\psi_{1i}
              \\=\langle\vec{v_2},\vec{v_1}\rangle
          $
          This proves the symmetric property.

          For any complex vector $\vec{v_1}$, $\langle \vec{v_1},\vec{v_1}\rangle=\sum_{i=0}^{n-1}\psi_{1i}^2$.
          For any complex number $\psi=a+bi$, we have $\psi^2=a^2+b^2\geq0$,

          so $\langle \vec{v_1},\vec{v_1}\rangle=\sum_{i=0}^{n-1}\psi_{1i}^2\geq0$ and $\vec{v_1}$ is a complex vector, so $\vec{v_1}\neq0$.

          This proves the positive definite property.

          So, it satisfies all the properties of an inner product.
\end{enumerate}
\section*{EXERCISE 2: THE TENSOR PRODUCT}
\begin{enumerate}
    \item
          $\ket{0}_A\otimes\ket{1}_B
              \\=\left(\begin{array}{c}1\\0\end{array}\right)\otimes\left(\begin{array}{c}0\\1\end{array}\right)
              \\=\left(\begin{array}{c}1\cdot{\left(\begin{array}{c}0\\1\end{array}\right)}\\0\cdot{\left(\begin{array}{c}0\\1\end{array}\right)}\end{array}\right)
              \\=\left(\begin{array}{c}1\times0\\1\times1\\0\times0\\0\times1\end{array}\right)
              \\=\left(\begin{array}{c}0\\1\\0\\0\end{array}\right)$
    \item
          $\ket{+}_A\otimes\ket{-}_B
              \\=\left(\begin{array}{c}\frac{1}{\sqrt{2}}\\\frac{1}{\sqrt{2}}\end{array}\right)\otimes\left(\begin{array}{c}\frac{1}{\sqrt{2}}\\-\frac{1}{\sqrt{2}}\end{array}\right)
              \\=\left(\begin{array}{c}\frac{1}{\sqrt{2}}\cdot{\left(\begin{array}{c}\frac{1}{\sqrt{2}}\\-\frac{1}{\sqrt{2}}\end{array}\right)}\\\frac{1}{\sqrt{2}}\cdot{\left(\begin{array}{c}\frac{1}{\sqrt{2}}\\-\frac{1}{\sqrt{2}}\end{array}\right)}\end{array}\right)
              \\=\left(\begin{array}{c}\frac{1}{\sqrt{2}}\times\frac{1}{\sqrt{2}}\\\frac{1}{\sqrt{2}}\times-\frac{1}{\sqrt{2}}\\\frac{1}{\sqrt{2}}\times\frac{1}{\sqrt{2}}\\\frac{1}{\sqrt{2}}\times-\frac{1}{\sqrt{2}}\end{array}\right)
              \\=\left(\begin{array}{c}\frac{1}{2}\\-\frac{1}{2}\\\frac{1}{2}\\-\frac{1}{2}\end{array}\right)
          $
    \item
          $\ket{0}_A\otimes\ket{-}_B
              \\=\left(\begin{array}{c}1\\0\end{array}\right)\otimes\left(\begin{array}{c}\frac{1}{\sqrt{2}}\\-\frac{1}{\sqrt{2}}\end{array}\right)
              \\=\left(\begin{array}{c}1\times\frac{1}{\sqrt{2}}\\1\times-\frac{1}{\sqrt{2}}\\0\times\frac{1}{\sqrt{2}}\\0\times-\frac{1}{\sqrt{2}}\end{array}\right)
              \\=\left(\begin{array}{c}\frac{1}{\sqrt{2}}\\-\frac{1}{\sqrt{2}}\\0\\0\end{array}\right)
              \\=\left(\begin{array}{c}\frac{\sqrt{2}}{2}\\-\frac{\sqrt{2}}{2}\\0\\0\end{array}\right)$
    \item $\ket{1}_A\otimes\ket{1}_B
              \\=\left(\begin{array}{c}0\\1\end{array}\right)\otimes\left(\begin{array}{c}0\\1\end{array}\right)
              \\=\left(\begin{array}{c}0\cdot{\left(\begin{array}{c}0\\1\end{array}\right)}\\1\cdot{\left(\begin{array}{c}0\\1\end{array}\right)}\end{array}\right)
              \\=\left(\begin{array}{c}0\times0\\0\times1\\1\times0\\1\times1\end{array}\right)
              \\=\left(\begin{array}{c}0\\0\\0\\1\end{array}\right)$
    \item We have
          $\ket{\Phi^+}
              \\=\frac{1}{\sqrt{2}}(\ket{0}_A\otimes\ket{1}_B+\ket{1}_A\otimes\ket{0}_B)
              \\=\frac{1}{\sqrt{2}}(\left(\begin{array}{c}1\\0\end{array}\right)\otimes\left(\begin{array}{c}0\\1\end{array}\right)+\left(\begin{array}{c}0\\1\end{array}\right)\otimes\left(\begin{array}{c}1\\0\end{array}\right))
              \\=\frac{1}{\sqrt{2}}(\left(\begin{array}{c}0\\1\\0\\0\end{array}\right)+\left(\begin{array}{c}0\\0\\1\\0\end{array}\right))
              \\=\frac{1}{\sqrt{2}}(\left(\begin{array}{c}0\\1\\1\\0\end{array}\right))
          $

          For $A=\left(\begin{array}{c}
                      a_0 \\a_1
                  \end{array}\right)$ and $B=\left(\begin{array}{c}
                      b_0 \\b_1
                  \end{array}\right)$,
        we have $A\otimes B=\left(\begin{array}{c}
            a_0b_0\\a_0b_1\\a_1b_0\\a_1b_1
        \end{array}\right)$.

        If $\ket{\Phi^+}$ can be written as $A\otimes B$, then
        
        $a_0b_0=0\\a_0b_1=1\\a_1b_0=1\\a_1b_1=0$.

        To make $a_0b_0=0$, either $a_0=0$ or $b_0=0$ should be true.

        If any of them is true, then $a_0b_1=1$ and $a_1b_0=1$ cannot be true in the same time.

        So $\ket{\Phi^+}$ can not be written as $A\otimes B$.
        \item We have $\ket{0}\ket{0}=\left(\begin{array}{c}
            1\\0\\0\\0
        \end{array}\right)$ and $\ket{1}\ket{1}=\left(\begin{array}{c}
            0\\0\\0\\1
        \end{array}\right)$.

        We also have
        $\ket{+}\ket{-}=\frac{1}{2}(\ket{0}\ket{0}-\ket{1}\ket{1})$

        and $\ket{-}\ket{+}=\frac{1}{2}(\ket{0}\ket{0}-\ket{1}\ket{1})$
        
        $\ket{\Phi^-}=\frac{1}{\sqrt{2}}(\ket{0}_A\otimes\ket{1}_B-\ket{1}_A\otimes\ket{0}_B)$
        
        So, $-\ket{\Phi^-}
        \\=\frac{1}{\sqrt{2}}(\ket{1}_A\otimes\ket{0}_B-\ket{0}_A\otimes\ket{1}_B)
        \\=\frac{1}{\sqrt{2}}(\left(\begin{array}{c}
            0\\0\\0\\0
        \end{array}\right)-\left(\begin{array}{c}
            0\\0\\0\\0
        \end{array}\right))
        \\=\left(\begin{array}{c}
            0\\0\\0\\0
        \end{array}\right)
        \\=\frac{1}{\sqrt{2}}[\frac{1}{2}(\ket{0}\ket{0}-\ket{1}\ket{1})-\frac{1}{2}(\ket{0}\ket{0}-\ket{1}\ket{1})]
        \\=\frac{1}{\sqrt{2}}(\ket{+}\ket{-}-\ket{-}\ket{+})
        % \\=\frac{1}{\sqrt{2}}[(\left(\begin{array}{c}
        %     1\\0\\0\\0
        % \end{array}\right)-\left(\begin{array}{c}
        %     0\\0\\0\\1
        % \end{array}\right))-(\left(\begin{array}{c}
        %     1\\0\\0\\0
        % \end{array}\right)-\left(\begin{array}{c}
        %     0\\0\\0\\1
        % \end{array}\right))]
        % \\==\frac{1}{\sqrt{2}}\times2\times(\ket{0}\ket{0}-\ket{1}\ket{1})
        % \\=\frac{2}{\sqrt{2}}[(\ket{0}+\ket{1})\otimes(\ket{0}-\ket{1})]
        % \\=
        $

        So $\ket{\Phi^-}$ in basis $\mathcal{B}_1$ is equal to $-\ket{\Phi^-}$ in basis $\mathcal{B}_2$.
        
\end{enumerate}
\begin{thebibliography}{99}
    \bibitem{ref1}HKUST Department of Mathematics.
    "Math111 Week 13-14 Lecture Notes."
    Hong Kong University of Science and Technology, n.d.,
    https://www.math.hkust.edu.hk/~mabfchen/Math111/Week13-14.pdf.
\end{thebibliography}
\end{document}