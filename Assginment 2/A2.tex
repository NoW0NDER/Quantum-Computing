\documentclass{article}

% Import necessary packages
\usepackage{amsmath}
\usepackage{amssymb}
\usepackage{graphicx}
\usepackage{hyperref}
\usepackage{listings}
\usepackage{xcolor}
\usepackage{braket}
% Set page margins
\usepackage[margin=1in]{geometry}

% Set code block style
\lstset{
    basicstyle=\ttfamily,
    keywordstyle=\color{blue},
    commentstyle=\color{gray},
    stringstyle=\color{purple},
    showstringspaces=false,
    breaklines=true,
    breakatwhitespace=true,
    captionpos=b,
    frame=single,
    numbers=left,
    numbersep=5pt,
    numberstyle=\tiny\color{gray},
    backgroundcolor=\color{lightgray!20},
    rulecolor=\color{black},
    tabsize=4
}

% Set title and author
\title{Assignment 2}
\author{Wang Dingrui}

\begin{document}

\maketitle

\section{EXERCISE 1: THE PARTIAL TRACE}

We saw in the lectures that given a multi-partite state, we obtain the state of a subsystem by applying the partial trace to the other systems.

\begin{enumerate}
    \item Compute the trace of the following states by applying the trace formula $ Tr(A)=\sum_i\bra{i}A\ket{i} $ T r(A) from the lectures:
          \begin{enumerate}
              \item $\xi = \frac{1}{2}(\ket{0}\bra{0} - i\ket{0}\bra{1} + i\ket{1}\bra{0} + \ket{1}\bra{1})$
              \item $\Lambda = \frac{I}{3}I + \frac{1}{6}(\ket{0}\bra{1} + \ket{1}\bra{0})$ (here, $I$ is the identity matrix).

                    Which of the states is normalized correctly? (*) (4 points)
          \end{enumerate}
          Answer: $\xi = \frac{1}{2}(\ket{0}\bra{0} - i\ket{0}\bra{1} + i\ket{1}\bra{0} + \ket{1}\bra{1})
              \\= \frac{1}{2}(\begin{bmatrix}
                  1 & 0 \\
                  0 & 0
              \end{bmatrix} - i\begin{bmatrix}
                  0 & 1 \\
                  0 & 0
              \end{bmatrix} + i\begin{bmatrix}
                  0 & 0 \\
                  1 & 0
              \end{bmatrix} + \begin{bmatrix}
                  0 & 0 \\
                  0 & 1
              \end{bmatrix})
              \\= \frac{1}{2}(\begin{bmatrix}
                  1 & -i \\
                  i & 1
              \end{bmatrix})
              \\= \begin{bmatrix}
                  \frac{1}{2} & -\frac{i}{2} \\
                  \frac{i}{2} & \frac{1}{2}
              \end{bmatrix}$

          $Tr(\xi) = \ket{0}\xi\bra{0}+\ket{1}\xi\bra{1}
              \\= \begin{bmatrix}
                  1 & 0
              \end{bmatrix}\begin{bmatrix}
                  \frac{1}{2} & -\frac{i}{2} \\
                  \frac{i}{2} & \frac{1}{2}
              \end{bmatrix}\begin{bmatrix}
                  1 \\
                  0
              \end{bmatrix}+\begin{bmatrix}
                  0 & 1
              \end{bmatrix}\begin{bmatrix}
                  \frac{1}{2} & -\frac{i}{2} \\
                  \frac{i}{2} & \frac{1}{2}
              \end{bmatrix}\begin{bmatrix}
                  0 \\
                  1
              \end{bmatrix}
              \\= \begin{bmatrix}
                  1 & 0
              \end{bmatrix}\begin{bmatrix}
                  \frac{1}{2} \\
                  \frac{i}{2}
              \end{bmatrix}+\begin{bmatrix}
                  0 & 1
              \end{bmatrix}\begin{bmatrix}
                  -\frac{i}{2} \\
                  \frac{1}{2}
              \end{bmatrix}
              \\= \frac{1}{2}+\frac{1}{2}
              \\= 1$

          $\Lambda = \frac{I}{3}I + \frac{1}{6}(\ket{0}\bra{1} + \ket{1}\bra{0})
              \\=\frac{I}{3}+\frac{1}{6}(\begin{bmatrix}
                  1 & 0 \\
              \end{bmatrix}\begin{bmatrix}
                  0 \\
                  1
              \end{bmatrix}+\begin{bmatrix}+\begin{bmatrix}
                       0 & 1
                   \end{bmatrix}\begin{bmatrix}
                                    1 \\
                                    0
                                \end{bmatrix})
                  \\=\frac{I}{3}+\frac{1}{6}(\begin{bmatrix}
                      0 & 0 \\
                      1 & 0
                  \end{bmatrix}+\begin{bmatrix}
                      0 & 1 \\
                      0 & 0
                  \end{bmatrix})
                  \\=\frac{I}{3}+\frac{1}{6}(\begin{bmatrix}
                      0 & 1 \\
                      1 & 0
                  \end{bmatrix})
                  \\=\begin{bmatrix}
                      0           & \frac{1}{3} \\
                      \frac{1}{3} & 0
                  \end{bmatrix}+\begin{bmatrix}
                      0           & \frac{1}{6} \\
                      \frac{1}{6} & 0
                  \end{bmatrix}
              \end{bmatrix}
              \\=\begin{bmatrix}
                  \frac{1}{3} & \frac{1}{6} \\
                  \frac{1}{6} & \frac{1}{3}
              \end{bmatrix}$
          $Tr(\Lambda) = \ket{0}\Lambda\bra{0}+\ket{1}\Lambda\bra{1}
              \\= \begin{bmatrix}
                  1 & 0
              \end{bmatrix}\begin{bmatrix}
                  \frac{1}{3} & \frac{1}{6} \\
                  \frac{1}{6} & \frac{1}{3}
              \end{bmatrix}\begin{bmatrix}
                  1 \\
                  0
              \end{bmatrix}+\begin{bmatrix}
                  0 & 1
              \end{bmatrix}\begin{bmatrix}
                  \frac{1}{3} & \frac{1}{6} \\
                  \frac{1}{6} & \frac{1}{3}
              \end{bmatrix}\begin{bmatrix}
                  0 \\
                  1
              \end{bmatrix}
              \\= \begin{bmatrix}
                  1 & 0
              \end{bmatrix}\begin{bmatrix}
                  \frac{1}{3} \\
                  \frac{1}{6}
              \end{bmatrix}+\begin{bmatrix}
                  0 & 1
              \end{bmatrix}\begin{bmatrix}
                  \frac{1}{6} \\
                  \frac{1}{3}
              \end{bmatrix}
              \\= \frac{1}{3}+\frac{1}{3}
              \\= \frac{2}{3}$

          So $\xi$ is normalized correctly.


    \item In the last assignment we found the probability to find a state $\ket{\gamma}$ in another state $\ket{\delta}$ to be $p = |\braket{\gamma|\delta}|^2$. Show here that this expression coincides with $\mathrm{Tr}(\gamma\delta)$, for $\gamma = \ket{\gamma}\bra{\gamma}$ and $\delta = \ket{\delta}\bra{\delta}$. Hint: Apply the trace formula $\mathrm{Tr}(A) = \sum_i \braket{i|A|i}$ from the lectures. Use a suitable basis of your choice. (4 points)

          Answer: $\mathrm{Tr}(\gamma\delta)
              \\= \mathrm{Tr}(\ket{\gamma}\bra{\gamma}\ket{\delta}\bra{\delta})
              \\= \mathrm{Tr}(\ket{\gamma}\braket{\gamma|\delta}\bra{\delta})
              \\= \mathrm{Tr}(\braket{\gamma|\delta}\ket{\gamma}\bra{\delta})
              \\= \braket{\gamma|\delta}\mathrm{Tr}(\ket{\gamma}\bra{\delta})
              \\= \braket{\gamma|\delta}\sum_i\braket{i|\ket{\gamma}\bra{\delta}|i}
              \\= \braket{\gamma|\delta}(\bra{0}\ket{\gamma}\bra{\delta}\ket{0}+\bra{1}\ket{\gamma}\bra{\delta}\ket{1})
              \\= \braket{\gamma|\delta}(\braket{0|\gamma}\braket{\delta|0}+\braket{1|\gamma}\braket{\delta|1})
              \\= \braket{\gamma|\delta}(\braket{\gamma|0}\braket{0|\delta}+\braket{\gamma|1}\braket{1|\delta})
              \\= \braket{\gamma|\delta}(\braket{\gamma|\delta})
              \\= |\braket{\gamma|\delta}|^2$

          So, $\mathrm{Tr}(\gamma\delta) = |\braket{\gamma|\delta}|^2 = p$.

    \item Consider the bipartite state $\ket{\phi}_{AB} = \frac{1}{\sqrt{3}}(\ket{00}_{AB} + i\ket{01}_{AB} - \ket{11}_{AB})$. Write the density operator $\rho_{AB} = \ket{\phi}_{AB}\bra{\phi}_{AB}$ explicitly in matrix form. (*) (4 points)
\end{enumerate}



\end{document}
